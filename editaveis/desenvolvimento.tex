\chapter[Componente de Hipervideo Orientado a Anotações]{Componente de Hipervideo Orientado a Anotações}

Este capítulo aborda questões relativas ao desenvolvimento do sistema. Para compreensão adequada do desenvolvimento do sistema, este capítulo foi divido em quatro seções com o objetivo de apresentar inicialmente as tecnologias utilizadas e a plataforma de videos interativos na qual o componente será integrado, em seguida o módulo de criação de anotações, logo após o modulo de visualização de Hipervídeos e por último como o componente será integrado à plataforma.

\section{Sistema de Videos Interativos}

Conforme as teorias apresentadas, os sistemas de videos interativos são sistemas que permitem certo grau de interação com o conteúdo audiovisual. De modo a atingir o objetivo de desenvolver um componente de hipervídeo proposto neste trabalho, foi definida uma plataforma para integração na qual desse possibilidade que um estudo sobre em quais termos este componente deveria ser construido.

A plataforma definida como objeto de estudo deste trabalho foi a \textit{Hypervídeos} \cite{arthurtcc}. Esta escolha foi feita levando em consideração que o projeto encontra-se ativo no Github\footnote{Ferramenta de centralização de código e controle de versão. Pode ser acessado pela URL www.github.com na data de publicação deste trabalho.}, além disso o projeto possui um ambiente de integração contínua com testes automatizados e controle de \textit{build\footnote{Versão do software pronta para distribuição.}}, licença GPL\footnote{GPL significa General Public License, uma licença que permite que o código seja utilizado e compartilhado, também garante que qualquer produto derivado seja distribuido sob a mesma licença.}, foi construido com base em tecnologias para Web (\textit{Meteor e Polymer}) e teve o desenvolvimento acompanhado de forma parcial pelo autor deste trabalho.

O \textit{framework} \textit{Meteor} é voltado para web, agrega recursos de reatividade e executa código \textit{Javascript}. Já o \textit{Polymer} consiste em uma biblioteca que permite a criação de componentes web para fins específicos.

Além dos motivos citados anteriormente, outro motivo para adoção desta plataforma como objeto de estudo foi a identificação das seguintes fraquezas: o componente de reprodução de video não suporta conteúdo hipermidiático, o que pode ser visto na figura xx; também a tela de construção de subvideos, que possibilita a criação da interatividade do conteúdo, não favorece o autor do curso no sentido que apresenta apenas a estrutura de decisão ao final de cada video, mostrado na figura xx.


\section{Módulo de Criação Ativa de Anotações}

\section{Módulo de Visualização de Hipervídeos}

\section{Integração com a plataforma Hypervídeos}
