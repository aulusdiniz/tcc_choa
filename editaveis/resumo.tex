\begin{resumo}
Não é de hoje que as dificuldades relacionadas à aprendizagem são questionadas pelos profissionais da área. Diversos pesquisadores desenvolveram metodologias buscando reduzir estas dificuldades enfrentadas por alunos e professores, inclusive algumas bem-sucedidas. Por outro lado, o desenvolvimento tecnológico tem agregado diversos elementos novos a esse contexto. A aplicação de jogos digitais educativos, ambientes de tutoria virtual, vídeos interativos e outros recursos advindos da evolução tecnológica no contexto educacional de forma adequada, tem se mostrado como forte aliado no momento de minimizar os problemas de aprendizado e também demonstra o interesse dos profissionais por esses tipos de soluções. Porém, o número de ferramentas disponíveis com o intuito de auxiliar na produção e utilização desses tipos de materiais é de difícil adaptação e em geral são limitadas pelos recursos oferecidos pelos desenvolvedores. Nesse contexto, este trabalho é proposto um componente web que agregue  características de hipermídias a vídeos, tornando-os em hipervídeos, facilitando tanto a produção de conteúdo interativo quanto a reprodução, visando manter um grau relativamente baixo de complexidade com relação à necessidade de adaptação das funcionalidades e evolução do sistema. Através deste trabalho pode-se concluir que a construção de um componente de hipervídeo para web que atenda as características propostas é possível.
 \vspace{\onelineskip}
    
 \noindent
 \textbf{Palavras-chaves}: teorias de aprendizagem. aprendizagem multimídia. sistemas de hipermídia adaptativa. componente orientado a anotações. engenharia de software.
\end{resumo}
