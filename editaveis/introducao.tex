\chapter[Introdução]{Introdução}

Este capítulo apresenta a contextualização, o problema de pesquisa, a justificativa, os objetivos, a metodologia de pesquisa utilizada e a organização deste trabalho com o objetivo de situar o leitor sobre o trabalho desenvolvido.

\section{Contextualização}

Uma parcela considerável de alunos obtém desempenho inferior ao desejado nos cursos de exatas, principalmente os cursos que exigem alta carga de conhecimentos matemáticos como matemática, física e engenharia. Quando os cursos de engenharia são tomados como exemplo, é possível verificar um alto índice de reprovação nas disciplinas de tronco comum além de uma alta taxa de evasão anual que fica em torno de 20\% \cite{fragelli2012summae}.

Esses índices decorrem de vários fatores, como uma preparação fraca durante o ensino médio, visão sobre o mercado profissional, características particulares das instituições de ensino, entre outros. Atualmente os educadores buscam diversas abordagens para tentar sanar esse problema relacionado à aprendizagem, trazendo elementos que facilitem este processo e motivem os estudantes \cite{fragelliplaycalculo}. 

Um exemplo é a utilização de recursos de vídeo, o que não é novidade quando se fala em educação, alguns professores aplicam como ferramenta para complementar o aprendizado e motivar. Um estudo realizado indicou que aproximadamente 84\% dos professores utilizam vídeos ao ministrar aulas, isto evidencia um interesse por parte desses profissionais, porém, apenas 17\% desses professores dão prioridade à este recurso. O uso de vídeos também é popular na modalidade de ensino a distância (EAD) \cite{vicentini2008uso}. 

Também podem ser citados outros exemplos de abordagens utilizadas para reduzir as barreiras do processo de aprendizagem como a aplicação de aulas interativas, uso de jogos educativos, e metodologias de ensino \cite{fragelli2012osama, fragelli2011batalha, fragelliplaycalculo}.

Este tipo de abordagem é relativamente recente e vem se tornando mais popular quanto mais professores fazem aplicação prática. O número crescente de professores adeptos à essas tecnologias também é devido as novas possibilidade que tecnologias como as hipermídias, câmeras portáteis, vídeos online e outras propiciaram.

Vicentini e Domingues (2008) mostram em seu estudo que 40\% dos professores que lançam mão de vídeos como material didático acessam canais como \textit{Youtube} \footnote{Site de compartilhamento de vídeos com recursos multimídia da Google Inc., pode ser acessado em www.youtube.com} ou semelhantes. Isso torna proeminente a necessidade de materiais compartilhados e as vezes com recursos extra, como links para outros materiais, comentários sobre trechos do vídeo ou tradução por meio de legendas \cite{vicentini2008uso}.

Partindo disto então foi que surgiu a motivação para a realização deste trabalho que propõe galgar um passo além nessa jornada para meios de aprendizagem mais eficientes do ponto de vista de alunos e professores. No trabalho é proposto a criação de um componente para sistemas web que permita que professores criem conteúdo audiovisual interativo de forma simples e com possibilidade de expansão de funcionalidades, para aqueles mais técnicos. Para tal, foi visado que o componente integrasse uma plataforma adaptativa de vídeos interativos que tem como foco a área do ensino, permitindo que professores aproveitem os espaços de compartilhamento de material em conjunto com elementos chamados hipervídeos.

\section{Problema de Pesquisa}

Com base no contexto apresentado, este trabalho pretende verificar se a construção de um componente web de hipervídeo orientado a anotações voltado para o ensino é factível.

\section{Justificativa}

Não há dúvidas que o uso adequado de recursos multimídia com foco na aprendizagem aumenta o desempenho médio dos alunos. Isso se deve à flexibilidade que a introdução do meio digital neste contexto cria, permitindo uma experiência de aprendizagem mais rica, que melhor se relaciona com o cognitivo do aluno \cite{moreno2000, zhang2005}.

A cada dia as pessoas ficam mais imersas em informações por meio da tecnologia. As tecnologias da informação e comunicação (TIC) estão se tornando cada vez mais interativas e parte integrante do cotidiano, contudo esse efeito não acompanha ao mesmo passo todas as áreas importantes da vida das pessoas, como a educação.

Ainda há uma grande defasagem com relação à aplicação dessas tecnologias na área da aprendizagem. No geral, para que professores apliquem de forma adequada as TICs é requerido um nível técnico alto para a elaboração de ferramentas que suportem a criação desses conteúdos, de modo que objeto de aprendizagem adapta-se aos perfis dos estudantes, o que limita severamente o desenvolvimento.

Existe uma discrepância grande quando se compara a utilização dessas tecnologias para o entretenimento em relação ao uso para educação. Foi visando reduzir essa discrepância que encontra-se a justificativa deste trabalho. Pretende-se fomentar a criação de conteúdo educacional baseado em hipervídeos assim como já acontece no âmbito do entretenimento.

Essa justificativa é válida pois muitos dos sistemas de hipervídeos existentes não possuem foco educativo, além de contar com soluções técnicas de difícil manutenção e evolução \cite{sadallah2012}.

\section{Objetivos}

Esta seção apresenta o objetivo geral da pesquisa e cada um dos objetivos específicos que precisam ser atingidos para alcançar a completude do trabalho.

\subsection{Objetivo Geral}

O objetivo geral deste trabalho é a construção de um componente web de hipervídeo orientado a anotações, multipedagógico, passível de evolução por meio de \textit{plug-ins} de componentes de anotação e integração com uma plataforma de vídeos interativos.

\subsection{Objetivos Específicos}

\begin{itemize}
	\item Construção do módulo de criação de anotações.
	\item Construção do módulo de reprodução de hipervídeos.
	\item Integração com plataforma de vídeos interativos.
\end{itemize}

\section{Metodologia}

Em alinhamento com os objetivos declarados, foi utilizado o método de pesquisa exploratória visando compreender melhor as linhas de pesquisa que serviram como aporte teórico para a modelagem do domínio de conhecimento. A primeira linha de pesquisa que recebeu enfoque foi a de teorias de aprendizagem, destacando as teorias de Skinner e Ausubel. Em seguida foi realizado um estudo sobre as teorias de sistemas de hipermídias adaptativas e aprendizagem multimídia. Por ultimo foi realizado um levantamento sobre as pesquisas e teorias de hipervídeos com ênfase nos métodos atuais de aplicação dessa tecnologia e tópicos relevantes de engenharia de software que se aplicam à este trabalho.

O processo de desenvolvimento do trabalho foi baseado em metodologias ágeis visando trazer o maior valor ao produto por unidade de tempo. Para aplicar os métodos ágeis, algumas práticas do \textit{Scrum} foram adotadas. Com relação ao desenvolvimento do sistema, foi utilizado o padrão arquitetural MVC, levando em consideração também que será lançado mão de recursos de reutilização de software para permitir uma flexibilidade na evolução do componente.

No primeiro momento, buscou-se compreender melhor o problema através da identificação de trabalhos científicos que pudesse esclarecer as causas, medidas tomadas e resultados, quais tecnologias foram empregadas nas tentativas de sanar o problema, para que então fosse possível determinar qual seria a abordagem deste trabalho.

Após a fase inicial, foi definido o escopo do trabalho com base nas necessidades apontadas e no recurso de tempo disponível para a execução. Também foi estudado e definido as tecnologias que foram empregadas neste trabalho.

Para validação da proposta, foi determinada uma plataforma onde o componente foi integrado, ao passo que acontecia o refinamento e documentação do trabalho desenvolvido, convergindo na elaboração deste TCC.

\section{Organização do Trabalho}

Este trabalho está dividido em quatro capítulos, sendo este de introdução o primeiro que traz a contextualização, problema de pesquisa, justificativa, objetivos, metodologia e organização do trabalho.

O capítulo 2 apresenta o referencial teórico utilizado e contém uma seção para explicar sobre as teorias de aprendizagem, sistemas de hipermídia adaptativa, aprendizagem multimídia, hipervídeos e engenharia de software.

O capítulo 3 contém os resultados da pesquisa, apresentando de que modo o sistema foi idealizado e como foi construído, apresenta uma seção para explicar sobre a plataforma que serve de aporte para o componente de hipervídeo, o módulo de criação de anotações, módulo de reprodução de hipervídeos e como foi realizada a integração com a plataforma.

No capítulo 4 foram apresentadas as considerações parciais do trabalho, proposta futura e cronograma para TCC 2.