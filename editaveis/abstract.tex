\begin{resumo}[Abstract]
 \begin{otherlanguage*}{english}
   It's not modern that difficulties related to learning are questioned by professionals. Several researchers have developed methodologies seeking to reduce these difficulties faced by students and teachers, including some successful. On the other hand, technological development has added several new elements to this context. The application of educational digital games, virtual tutoring environments, interactive videos and other resources coming from technological developments at the educational context in a properly way, has proven to be a strong ally at the time to minimize the learning problems and also demonstrates the professional interest for these types of solutions. But the number of tools available in order to assist the production and use of these types of materials is reduced, difficult to adapt and are usually limited by the resources offered by developers. Therefore in this work we propose a web component that adds features of hypermedia to videos, making them into hypervideos, facilitating both the generation of interactive content and playback, aiming to maintain a relatively low degree of complexity regarding the need to adapt the features and evolution of the system. Through this study it can be concluded that the construction of a hypervideo web component that meets the characteristics proposed is possible.

   \vspace{\onelineskip}
 
   \noindent 
   \textbf{Key-words}: learning theories. multimedia learning. adaptive hypermedia systems. component annotation-based. software engineering.
 \end{otherlanguage*}
\end{resumo}
